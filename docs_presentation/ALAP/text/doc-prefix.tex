%%% Doc-Prefix


\pagenumbering{roman}

\renewcommand{\abstractname}{Resumen}

\begin{abstract}
\setlength{\parindent}{2pt}
\noindent
En América Latina y el Caribe la demanda de datos a nivel sub-nacional es creciente, tanto como herramienta para la aplicación de distintos planes de desarrollo como para la asignación de recursos. Estudios recientes en los países desarrollados apuntan a señalar que las estimaciones de mortalidad en áreas menores tienden a ser diferenciales, encontrándose contrastes en la esperanza de vida al nacer entre distintas sub-regiones y/o grupos sociales, idea que discrepa, en parte, con la observada declinación en la variabilidad de la edad a la defunción desde mediados del siglo XX en países desarrollados, que sugiere que la mayoría de las muertes se concentran en una edad cada vez más estrecha y a la vez, más estable a medida que la mortalidad desciende. Considerando que no hay antecedentes en la temática para el caso argentino, en artículo se propone realizar estimaciones de estructura y niveles mortalidad para áreas pequeñas (sub-provinciales -departamentos-) en Argentina (provincias seleccionadas) durante el periodo 1980-2015. Para ello se propone combinar métodos indirectos de estimación demográfica y estadísticos. El plan de trabajo permitirá construir un insumo tanto para la elaboración de políticas públicas como para profundizaren los diferenciales de la mortalidad en las zonas geográficas seleccionadas. 
\\\\
Palabras clave: Mortalidad; Argentina; Áreas menores.
\end{abstract}


\renewcommand{\abstractname}{Abstract}
\noindent
\begin{abstract}
\noindent
En América Latina y el Caribe la demanda de datos a nivel sub-nacional es creciente, tanto como herramienta para la aplicación de distintos planes de desarrollo como para la asignación de recursos. 
Estudios recientes en los países desarrollados apuntan a señalar que las estimaciones de mortalidad en áreas menores tienden a ser diferenciales, encontrándose contrastes en la esperanza de vida al nacer entre distintas sub-regiones y/o grupos sociales, idea que discrepa, en parte, con la observada declinación en la variabilidad de la edad a la defunción desde mediados del siglo XX en países desarrollados, que sugiere que la mayoría de las muertes se concentran en una edad cada vez más estrecha y a la vez, más estable a medida que la mortalidad desciende. 
Considerando que no hay antecedentes en la temática para el caso argentino, en artículo se propone realizar estimaciones de estructura y niveles mortalidad para áreas pequeñas (sub-provinciales -departamentos-) en Argentina (provincias seleccionadas) durante el periodo 1980-2015. Para ello se propone combinar métodos indirectos de estimación demográfica y estadísticos. El plan de trabajo permitirá construir un insumo tanto para la elaboración de políticas públicas como para profundizaren los diferenciales de la mortalidad en las zonas geográficas seleccionadas.
\\\\
Keywords: Mortalidad; Argentina; Áreas menores.
\end{abstract}



{
\setcounter{tocdepth}{5}
\tableofcontents
}

\clearpage\pagenumbering{arabic}\setcounter{page}{1}